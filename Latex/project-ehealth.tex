% !TEX encoding = UTF-8
%Koma article
\documentclass[fontsize=12pt,paper=letter,twoside]{scrartcl}
\usepackage{graphicx}
\usepackage{multirow}
\usepackage{mathtools}

%Standard Pre-amble
\input{sty/defns.tex}

% Set the header
\ihead[]{\small EECS4312 eHealth Project}
\title{EECS4312 eHealth Project}

%Useful definitions
%\newcommand{\mv}[1]{\textit{m\_#1}}
%\newcommand{\cv}[1]{\textit{c\_#1}}
%\newcommand{\degree}[1]{^{\circ}\mathrm{#1}}
%\newcommand{\comment}[1]{{\footnotesize \quad\texttt{--}\textrm{#1}}}

%%%%%%%%%%%%Enter your names here%%%%%%%%
\author{{Siraj Rauff (cse23188@cse.yorku.ca)}
\and {Skyler Layne (cse23170@cse.yorku.ca)}
}
%%%%%%%%%%%%%%%%%%%%%%%%%%%%%%%%

\date{\today} % Display a given date or no date

\begin{document}
\maketitle

\noindent You may work on your own or in a team of no more than two students. \textbf{Submit only one document under one Prism account.} 

\bigskip
\noindent \textbf{Prism account used for submission}: cse23188

\bigskip\noindent
Keep track of your revisions in the table below.

\section*{Revisions}
%%%%%%%%%%%%Table of revisions%%%%%%%%
\begin{tabular}{|l|l|p{3in}|}
\hline
Date & Revision& Description \\ 
\hline

\hl{date please}

& 1.0       
& Initial requirements document\\ 
\hline
\end{tabular}
%%%%%%%%%%%%%%%%%%%%%%%%%%%%%%%%

\newpage

\vspace*{2in}
\begin{center}
\huge{\textbf{Requirements Document}:\\ for Patient care eHealth System}
\end{center}

\newpage

%%%%%%%%%%%%%%%%%%%%%%%%%%%%%%%
\tableofcontents
\listoffigures
\listoftables
\newpage

%%%%Rest of your document goes here%%%%%%%%%%%%%%%%%%%

\section{System Overview}

The System Under Development (SUD) is a computer system to create and manage health prescription records for Ontario.

This requirements document is specifically for prescription management. The purpose of the eHealth Patient care System is to maintain physicians, medications, patients, and patient prescriptions. The system will also control the undesirable interactions between medications, that is when two medications conflict in some way with one another. Only specialist physicians should be allowed to prescribe undesirable interactions while general physicians should be allowed to prescribe medications, as long as they do not create undesirable interactions. 

%\begin{figure}[!htb]
%\begin{center}
%\includegraphics[width=.4\textwidth]{pics/isolette.png}
%\end{center}
%\caption{Isolette}
%\label{fig:isolette}
%\end{figure}

\newpage


\section{Context Diagram}
 
The System Under Description (SUD) is a computer \emph{controller} to keep track of the physicians, patients, patient prescriptions, medications, and medication interactions. The monitored variables and controlled variables for this computer system can be found in Table~\ref{table:monitored} and Table~\ref{tbl:cv} respectively.

The system must keep track of abstract state which isn't available to the user. For a list of the abstract states within the controller see Figure~\ref{abs-state}.

\begin{figure}[htb]
\begin{center}
\includegraphics[width=0.8\textwidth]{pics/ContextDiagramExpanded.pdf}
\end{center}
\caption{Context Diagram}
\label{fig:modes}
\end{figure}

\newpage

\section{Goals}

The high-level goals (G) of the system are:

\begin{mylist}
\item G1 --- The user should be able to add physicians of type \{generalist, specialist\} with unique IDs
\item G2 --- The user should be able to add patients with unique IDs
\item G3 --- The user should be able to add medications with unique names and IDs and a safe dosage range
\item G4 --- The user should be able to add prescriptions with a unique ID between a doctor and patient
\item G5 --- The user should be able to add interactions between medications
\item G6 --- Physicians should be able to prescribe medications with specific dosages, so long as the dosage is within the safe range
\item G7 --- A Physician must be a specialist to prescribe a medication to a patient when it would create a dangerous interaction with a different already prescribed medication - regardless of the prescription or physician the other medication exists in
\item G8 --- The system should be able to display a report of all patients prescribed a particular medication
\item G9 --- The system should be able to display a dangerous prescriptions report that lists all patients with dangerous prescriptions along with the particular medications prescribed to them that interact dangeriously

\end{mylist}

\newpage

\section{Monitored Events}

The monitored events are those which come through the user interface. The following monitored events will be available to the user.

\begin{table}[h]
\centering
\begin{tabular}{|l|l|}
\hline
Name & Interpretation \\ \hline

\begin{tabular}[c]{@{}l@{}}add\_physician(id: ID\_MD; \\ name: NAME; kind: PHYSICIAN\_TYPE)\end{tabular}  & Add a Physician to the system \\ \hline

add\_patient(id: ID\_PT; name: NAME) & Add Patient to the system \\ \hline

\begin{tabular}[c]{@{}l@{}}add\_medication(id: ID\_MN; \\ medicine: MEDICATION)\end{tabular} & Add a medication to the system \\ \hline

add\_interaction(id1:ID\_MN;id2:ID\_MN) & \begin{tabular}[c]{@{}l@{}}Add an interaction between\\ two medications\end{tabular} \\ \hline

\begin{tabular}[c]{@{}l@{}}new\_prescription(id: ID\_RX; \\ doctor: ID\_MD; patient: ID\_PT)\end{tabular} & \begin{tabular}[c]{@{}l@{}}Add a new prescription to \\ the system\end{tabular} \\ \hline

\begin{tabular}[c]{@{}l@{}}add\_medicine(id: ID\_RX; \\ medicine:ID\_MN; dose: VALUE)\end{tabular} & Add a medicine to a prescription \\ \hline

\begin{tabular}[c]{@{}l@{}}remove\_medicine(id: ID\_RX; \\ medicine:ID\_MN)\end{tabular} & \begin{tabular}[c]{@{}l@{}}Remove a medication from\\ 
a prescription\end{tabular} \\ \hline

prescriptions\_q(medication\_id: ID\_MN) & \begin{tabular}[c]{@{}l@{}}List all patients prescribed\\ the medication\end{tabular} \\ \hline

dpr\_q & \begin{tabular}[c]{@{}l@{}}Print out a dangerous prescription report\end{tabular}  \\ \hline

\end{tabular}
\caption{Monitored Events}
\label{table:monitored}
\end{table}

\begin{table}[h]
\centering
\begin{tabular}{|l|l|l|}
\hline
Name & Type & Interpretation \\ \hline
PHYSICIAN\_TYPE & \{gn, sp\} & \begin{tabular}[c]{@{}l@{}}Generalist or Specialist\end{tabular} \\ \hline
MEDICATION & \begin{tabular}[c]{@{}l@{}}{[}name: NAME; kind: KIND; \\ low: VALUE; hi: VALUE{]}\end{tabular} & Name is unique. See following rows\\ \hline
KIND & \{pill, liquid\} & \begin{tabular}[c]{@{}l@{}}Pill or a liquid\end{tabular}                \\ \hline
DOSE & \{mg, cc\} & \begin{tabular}[c]{@{}l@{}}Milligrams \\ or Cubic Centimetres\end{tabular} \\ \hline
\end{tabular}
\caption{Monitored Types}
\label{table:monitored-types}
\end{table}

\newpage

\section{Controlled Variables}
The controlled variables represent what will be shown to the user.

\begin{table}[h]
\centering
\begin{tabular}{|l|l|l|}
\hline
Name          & Interpretation                                             & Abstract State \\ \hline
Physicians    & A list of all the Physicians currently within the system   & See table \ref{abs-state}  \\ \hline
Patients      & A list of all the Patients currently in the system         & See table \ref{abs-state}  \\ \hline
Medications   & A list of all the Medications currently within the system  & See table \ref{abs-state}  \\ \hline
Interactions  & A list of all the Interactions currently within the system & See table \ref{abs-state}  \\ \hline
Prescriptions & A list of all the Prescriptions within the system          & See table \ref{abs-state}  \\ \hline
Error         & A message displaying the highest priority error, or ok     & See table \ref{abs-state}  \\ \hline
\end{tabular}
\caption {Controlled Variables}
\label{tbl:cv}
\end{table}

\newpage
\section{Mode Diagram}

REQ\ref{R1} states The \emph{controller} shall operate in one of four modes: \emph{off}, \emph{init}, \emph{normal} and \emph{fail}, shown in Fig.~\ref{fig:sc}. As shown in the figure, the Isolette will begin in the \emph{off} mode, and will enter \emph{init} mode when the nurse flips \mv{sw} into the \emph{on} position. The Isolette will only be able to move from the \emph{init} mode ot the \emph{normal} mode if it is properly configured such that the desired range is valid and not overlapping with the alarm levels, and both the sensors and operator controls are working (see REQ\ref{R6}).
Once inside the \emph{normal} mode, the controller will move to the \emph{fail} mode if either the controls or sensor fails, as specified in REQ\ref{R5}, and will only return to the \emph{normal} mode once they are both working correctly (REQ\ref{R7}). In any of these states, if the nurse switches \mv{sw} to \emph{off} the controller will switch to the off mode (REQ\ref{R8}).

\begin{figure}[!htb]
\begin{mdframed}
\begin{center}
\includegraphics[width=.9\textwidth]{pics/mode-statechart.pdf}
\end{center}
\end{mdframed}
\caption{Statechart for the modes variable \cv{md}}
\label{fig:sc}
\end{figure}

\section{E/R-descriptions}

\subsection{Requirements Descriptions}
\reqm{REQ}
{The system will keep track of Physicians, Patients, Medications, Interactions, and Prescriptions\\}
{ See Table~\ref{tbl:cv} for list of abstract states.\\}
\label{R1}

\reqm{REQ}
{A generalist cannot add a medicine to a prescription if that medicine leads to a dangerous interaction. \\}
{ See Table~\ref{table:monitored} for adding medicine, and Table~\ref{tbl:cv} for dangerous interactions.\\}
\label{R2}

\reqm{REQ}
{An interaction cannot be added if a specialist has not prescribed at least one of the medications.\\}
{ See Table~\ref{table:monitored} for adding interaction, and Table~\ref{tbl:cv}.\\}
\label{R3}

\subsection{Environmental Descriptions}
\reqm{ENV}
{All input to the system will be constrained to the GUI grammer.\\}
{ See Table~\ref{table:monitored} for the list possible monitored events.\\}
\label{E1}

\reqm{ENV}
{A constraint by the grammar will ensure PHYSICIAN\_TYPE is either a generalist or a specialist.\\}
{ See Table~\ref{table:monitored-types} for PHYSICIAN\_TYPE.\\~}
\label{E2}
\noindent \textbf{Rationale}: The operator interface for the nurse is designed in such a way that the increments of the input temperatures from the control interface must change by whole numbers.

\reqm{ENV}
{The DOSE type will be constrained by the GUI grammar to be either mg, or cc.\\}
{ See Table~\ref{table:monitored-types} for DOSE.\\~}
\label{E3}

\reqm{ENV}
{The KIND type will be constrained by the GUI grammar to be either a pill or a liquid.\\}
{ See Table~\ref{table:monitored-types} for KIND.\\~}
\label{E4}

\newpage
%%%%%%%%%%%%%%%%%%%%%%%%%%%%
\section{Abstract variables needed for the Function Table}

\begin{figure}[!htb]
\begin{center}


\begin{tabular}{|l|l|l|}
\hline
Name & Interpredation                                                                                 & Purpose                                                                                                                         \\ \hline
mnid & The set of all medication ids                                                                  & Keep track of all the  medication ids in the system                                                                             \\ \hline
ptid & The set of all patient ids                                                                     & Keep track of all the patient ids in the system                                                                                 \\ \hline
mdid & The set of all doctor ids                                                                      & Keep track of all the doctor ids in the system                                                                                  \\ \hline
rxid & The set of all prescription ids                                                                & Keep track of all the prescription ids in the system                                                                            \\ \hline
mdpt & \begin{tabular}[c]{@{}l@{}}Relationship between\\ Doctor and Patient\end{tabular}              & Keep track of the doctor and the patient                                                                                        \\ \hline
rx   & \begin{tabular}[c]{@{}l@{}}Relation between doctor\\ patient and medication\end{tabular}       & \begin{tabular}[c]{@{}l@{}}Keep track of all the medical prescriptions between\\  doctor and patient in the system\end{tabular} \\ \hline
prs  & \begin{tabular}[c]{@{}l@{}}List of all prescriptions\\ including dosage\end{tabular}           & Keep track of all the prescriptions in the system                                                                               \\ \hline
di   & \begin{tabular}[c]{@{}l@{}}Set of dangerous \\ interactions between\\ medications\end{tabular} & \begin{tabular}[c]{@{}l@{}}Keep track of all the dangerous interactions in the\\ system\end{tabular}                            \\ \hline
gs   & The kind of doctor                                                                             & Keep track of all the type of doctor                                                                                            \\ \hline
dpi  & \begin{tabular}[c]{@{}l@{}}The dangerous\\  prescription report\end{tabular}                   & \begin{tabular}[c]{@{}l@{}}Notify dangerous interactions,\\ if they exist\end{tabular}                                          \\ \hline
\end{tabular}


\caption{Abstract Variables used in Function Tables}
\label{abs-state}
\end{center}
\end{figure}




\section{Function Tables}
%%%%%%%%%%%%%%%%%%%%%%%%%%%
\subsection{Function Table for Heat Control: \cv{hc}}
\begin{figure}[!htb]
\begin{center}
\begin{tabular}{|l|l|l|l|l|}
\hline
\multicolumn{4}{|l|}{Monitored Inputs \cv{md(i)}} & c\_hc(i) \\ \hline
\multicolumn{4}{|l|}{$i = 0$}& off \\ \hline
\multirow{4}{*}{$i\,\textgreater\,0$} & \multicolumn{3}{l|}{$c\_md(i) = off \lor \neg c1 \lor \neg c3$}& off      \\ \cline{2-5} 
& \multirow{3}{*}{$\neg c\_md(i) = off \land c1 \land c3$} & \multicolumn{2}{l|}{c2}                            & NC       \\ \cline{3-5} 
&                                                  & \multicolumn{2}{l|}{$m\_tm(i)\,\textless\,m\_dl(i)$}    & on       \\ \cline{3-5} 
& & \multicolumn{2}{l|}{$m\_tm(i)\,\textgreater\,m\_dh(i)$} & off      \\ \hline
\end{tabular}
\caption{Function Table for heat control: \cv{hc}}
\label{c_hc_ft}
\end{center}
\end{figure}

\newpage
%%%%%%%%%%%%%%%%%%%%%%%
\subsection{Function Table for Temperature Display: \cv{td}}
\begin{figure}[!htb]
\begin{center}
\begin{tabular}{|l|l|}
\hline
Monitored Inputs \mv{tm(i)}, \cv{md(i)} & \cv{td(i)} \\ \hline
$\cv{md(i)} = normal$         & \mv{tm(i)}    \\ \hline
$\neg \cv{md(i)} = normal$    & 0        \\ \hline
\end{tabular}

\caption{Function Table for Temperature Display: \cv{td}}
\label{c_td_ft}
\end{center}
\end{figure}

%%%%%%%%%%%%%%%%%%%%%%%
\subsection{Function Table for Mode: \cv{md}}
\begin{figure}[!htb]
\begin{center}
\begin{tabular}{|l|l|l|l|l|}
\hline
\multicolumn{4}{|l|}{Monitored Inputs \mv{sw(i)}, \cv{md(i-1)}} & \cv{md(i)}          \\ \hline
\multicolumn{4}{|l|}{$i=0$} & off    \\ \hline
\multirow{6}{*}{$i\,\textgreater\,0$} & \multicolumn{3}{l|}{\mv{sw(i)} = off}  & off    \\ \cline{2-5} 
                                & \multirow{5}{*}{\mv{sw(i)} = on} & \multicolumn{2}{l|}{$\cv{md(i-1)} = off$} & init  \\ \cline{3-5} 
                                &                              & \multirow{2}{*}{$\cv{md(i-1)} = normal \lor \cv{md(i-1)} = failed$} & $c1(i)$ & normal \\ \cline{4-5} 
                                &                              & & $\neg\,c1(i)$  & failed \\ \cline{3-5} 
                                &                              & \multirow{2}{*}{$\cv{md(i-1)} = init$}& $c4(i)$ & init \\ \cline{4-5} 
                                &                              &  & $\neg\,c4(i)$     & normal  \\\hline
\end{tabular}
\caption{Function Table for Mode: \cv{md}}
\label{c_md_ft}
\end{center}
\end{figure}

%%%%%%%%%%%%%%%%%%%%%%%
\subsection{Function Table for Messages: \cv{ms}}

\begin{figure}[!htb]
\begin{center}
\begin{tabular}{|l|l|}
\hline
Monitored Inputs \mv{al(i)}, \mv{ah(i)}, \mv{tm(i)} & \cv{ms(i)}         \\ \hline
$\neg\,c1(i)$ & invalid \\ \hline
$\neg\,c3(i)$ & config \\ \hline
$\mv{tm(i)} \,\textless\,\mv{al(i)}$ & low \\ \hline
$\mv{tm(i)}\,\textgreater\,\mv{ah(i)}$ & high \\ \hline
ELSE                                 & ok  \\ \hline
\end{tabular}
\caption{Function Table for Messages: \cv{ms}}
\label{c_ms_ft}
\end{center}
\end{figure}

\newpage
%%%%%%%%%%%%%%%%%%%%%%%
\subsection{Function Table for Alarm: \cv{al}}

\begin{figure}[!htb]
\begin{center}
\begin{tabular}{|l|l|l|l|l|}
\hline
\multicolumn{4}{|l|}{Monitored Inputs \mv{al(i)}, \mv{ah(i)}, \mv{tm(i)}} & \cv{al(i)}     \\ \hline
\multicolumn{4}{|l|}{$i = 0$} & off \\ \hline
\multirow{5}{*}{$i\,\textgreater\,0$} & \multirow{2}{*}{$\cv{al(i-1) = off}$} & \multicolumn{2}{l|}{$c7(i)$} & NC \\ \cline{3-5} 
                                &                                & \multicolumn{2}{l|}{$\neg c7(i)$} & on \\ \cline{2-5}
                                & \multirow{3}{*}{$\cv{al(i-1) = on}$} &   \multicolumn{2}{l|}{$c8(i)$} & NC \\ \cline{3-5} 
                                &                                & \multirow{2}{*}{$\neg c8(i)$} & $held\_for(i)$ & off  \\ \cline{4-5} 
                                &                                &                                    & $\neg held\_for(i) $ & on \\ \hline
                                
\end{tabular}
\caption{Function Table for Alarm: \cv{al}}
\label{c_al_ft}
\end{center}
\end{figure}

%%%%%%%%%%%%%%%%%%%%%%%%%%%%
\section{Validation}
\hl{todo}
\begin{figure}[!htb]
\begin{center}
\includegraphics[width=1\textwidth]{pics/top.png}
\end{center}
\caption{Validated Isolette}
\label{proofs}
\end{figure}


You must also provide and prove in PVS one important safety invariant for the heat control $\cv{hc}$ and one important safety invariant for the alarm control \cv{al}.

Include the PVS sources in the appendix to this document but summarize the proofs here (top.summary).

\newpage 
\section{Use Cases}

See Section A2 of \cite{REMH} for some use cases. The use cases need to be adapted to the revised descriptions of the previous sections of this document.

\section{Acceptance Tests}

In this section, the use cases have to be converted into precise acceptance tests (using the function table to describe pre/post conditions) to be run when the design and implementation are complete.

\section{Traceability}

Matrix to show which acceptance tests passed, and which R-descriptions they checked.


\section{Glossary}

The definition of important terms is placed in this section. You are not required to complete this.
%%%%%%%%%%%%%%%%%%%%%%%%%%%%%%%%%%%%%
\newpage
\bibliographystyle{plain}
\bibliography{ref}

\newpage
\appendix 

\section{Additional Requirements}

\reqm{REQ}
{In \emph{fail} mode, the controller shall only return to \emph{normal} mode if
\begin{mylist}
\item The sensor is working and
\item The operator controls are working
\end{mylist}~}
{In \emph{fail} mode the controller shall return to \emph{normal} mode when \mv{st} returns ``valid"~\\}
\label{R7}

\reqm{REQ}
{In any mode, the controller will transition to the \emph{off} mode if the nurse turns the switch off.~\\}
{The controller will transition to \emph{off} mode from any mode if \mv{sw} becomes \emph{off}\\~}
\label{R8}


\newpage
\section{Isolette PVS}
\begin{pvs}
isolette[delta:posreal]: THEORY
BEGIN

  %% Import timing resolution
  importing Time[delta]
  i: VAR DTIME

  %% TYPE declarations
  SWITCH: TYPE = {on, off}
  MSTATE: TYPE =  {valid, invalid}
  CONTROL: TYPE = {on, off}
  STATE: TYPE = {off, init, normal, failed}
  ERROR: TYPE = {ok, invalid, config, low, high}
  DISPLAY: TYPE = {i: nat | i = 0 OR (68 <= i AND i <= 105)}
  	CONTAINING 0
  ALARM: TYPE = {on, off}
  TM: TYPE+ = {r: real | r >= 68.0 AND r <= 105.0} CONTAINING 68.0
  DL: TYPE+ = {i: nat | i >= 97 AND i <= 99} CONTAINING 97
  DH: TYPE+ =  {i: nat | i >= 98 AND i <= 100} CONTAINING 98
  AL: TYPE+ = {i: nat | i >= 93 AND i <= 98} CONTAINING 93
  AH: TYPE+ = {i: nat | i >= 99 AND i <= 103} CONTAINING 99

  %% Monitored Variables
  m_tm: [DTIME -> TM]
  m_dl: [DTIME -> DL]
  m_dh: [DTIME -> DH]
  m_al: [DTIME -> AL]
  m_ah: [DTIME -> AH]
  m_st: [DTIME -> MSTATE]
  m_sw: [DTIME -> SWITCH]

  %% Controlled Variables
  c_hc: [DTIME -> CONTROL]
  c_td: [DTIME -> DISPLAY]
  c_al: [DTIME -> ALARM]
  c_md: [DTIME -> STATE]
  c_ms: [DTIME -> ERROR]
  
\end{pvs}
\newpage
\begin{pvs}

  al_on(i): bool = c_al(i) = on %% Alarm is on

  % General Function table conditions
  c1(i): bool = m_st(i) = valid
  c2(i): bool = m_dl(i) <= m_tm(i) <= m_dh(i)
  c3(i): bool = m_al(i) < m_dl(i) < m_dh(i) < m_ah(i)
  c4(i): bool = c1(i) AND c2(i) AND c3(i)
  c5(i): bool = m_tm(i) <= m_al(i) + 0.5
  c6(i): bool = m_tm(i) <= m_al(i) - 0.5
  c7(i): bool = c1(i) AND c3(i) AND m_al(i) < m_tm(i) < m_ah(i)
  c8(i): bool = NOT c1(i) OR NOT c3(i) OR c5(i) OR c6(i)
  held_for(i): bool = held_for(al_on, 10)(i)

  % Mode Function Table
  c_md_ft(i): bool =
  COND
    	i = 0 ->  c_md(i) = off,
	i > 0 ->
	COND
	  	m_sw(i) = off -> c_md(i) = off,
		m_sw(i) = on ->
		COND
			c_md(i-1) = off -> c_md(i) = init,
			c_md(i-1) = normal OR c_md(i-1) = failed ->
			COND
				NOT c1(i) -> c_md(i) = failed,
				c1(i) -> c_md(i) = normal
			ENDCOND,
			c_md(i-1) = init ->
			COND
				NOT c4(i) ->  c_md(i) = normal,
				c4(i) ->  c_md(i) = init
			 ENDCOND
		ENDCOND
	ENDCOND
  ENDCOND
  
\end{pvs}
\newpage
\begin{pvs}

  % Temperature Display Function Table
  c_td_ft(i): bool =
  COND
  	c_md(i) = normal -> c_td(i) = m_tm(i),
	NOT c_md(i) = normal -> c_td(i)= 0
  ENDCOND

  % Heat Control Function Table
  c_hc_ft(i): bool =
	    COND
		i = 0 -> c_hc(i) = off,
		i > 0 ->
		  COND
		    c_md(i) = off OR (NOT c1(i)) 
		      OR (NOT c3(i)) -> c_hc(i) = off,
  		  (NOT c_md(i) = off) AND c1(i) AND c3(i) ->
 		        COND
  			      c2(i) ->  c_hc(i) =  c_hc(i-1),
  				    m_tm(i) < m_dl(i) 
				      -> c_hc(i) = on,
  				    m_tm(i) > m_dh(i) 
				      -> c_hc(i) = off
		        ENDCOND
		    ENDCOND
	    ENDCOND

\end{pvs}
\newpage
\begin{pvs}

  % Alarm Function Table
  c_al_ft(i): bool =
  COND
	i = 0 ->  c_al(i) = off,
	i > 0 ->
	COND
		c_al(i-1) = off ->
		COND
			c7(i) ->  c_al(i) =  c_al(i-1),
			NOT c7(i) -> c_al(i) = on
		ENDCOND,
		c_al(i-1) = on ->
		COND
			c8(i) ->  c_al(i) =  c_al(i-1),
			NOT c8(i) ->
			COND
				held_for(i) -> c_al(i) = off,
				NOT held_for(i) -> c_al(i) = on
			ENDCOND
		 ENDCOND
	ENDCOND
  ENDCOND

  % Message Display Function Table
  c_ms_ft(i): bool =
	    IF m_st(i) = invalid THEN c_ms(i) = invalid
	    ELSIF NOT c3(i) THEN c_ms(i) = config
	    ELSIF m_tm(i) < m_al(i) THEN c_ms(i) = low
	    ELSIF m_tm(i) > m_ah(i) THEN c_ms(i) = high
	    ELSE c_ms(i) = ok
	    ENDIF


  % Isolette Specification
  isolette(i): bool = c_hc_ft(i) AND c_td_ft(i) AND c_al_ft(i)
  	AND c_md_ft(i) AND c_ms_ft(i)

\end{pvs}
\newpage
\begin{pvs}

  % Checks
  inv_hc(i): bool = NOT (c1(i) AND c3(i)) IMPLIES c_hc(i) = off
  inv_hc_holds: CONJECTURE (FORALL i: isolette(i)) =>
  				(FORALL i: inv_hc(i))

  inv_al(i): bool = i > 0 AND (
    (NOT al_on(i-1) AND NOT c7(i))
    OR (al_on(i-1) AND c8(i))
    OR (al_on(i-1) AND NOT c8(i) AND NOT held_for(i))
  ) IMPLIES c_al(i) = on
  inv_al_holds: CONJECTURE (FORALL i: i>0 IMPLIES isolette(i))
  				=> (FORALL i: i>0 IMPLIES inv_al(i))

END isolette
\end{pvs}


\end{document}  
