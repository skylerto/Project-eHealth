\section{E/R-descriptions}

\hl{To be Done}

\hl{Include the REQ descriptions provided in the write-up and add the next two most important REQ}. (You may include the remaining REQs in an appendix to this document, if you wish). Provide a rationale for each REQ.

\hl{Include the ENV descriptions provided in the write-up and add the next two most important ENV}. (You may include the rest in an appendix to this document, if you wish). Provide a rationale for each ENV.

\reqm{REQ}
{The \emph{controller} shall operate in one of four modes: \emph{off}, \emph{init}, \emph{normal} and \emph{fail}.\\}
{See statechart in Fig.~\ref{fig:sc}.}
\label{R1}

\reqm{REQ}
{In the \emph{normal} mode, the temperature controller shall maintain current temperature inside the Isolette within a set temperature range (the \emph{desired} range).\\}
{The \emph{desired} temperature range is $\mv{dl} \upto \mv{dh}$. If the current temperature \mv{tm} is outside this range, the controller shall turn the heater on or off via the controlled variable \mv{hc} to maintain the desired state.\smallskip}
\label{R2}

\reqm{REQ}
{In \emph{normal} mode, the controller shall activate an alarm whenever 

\begin{mylist}
\item the current temperature falls outside the \emph{alarm} temperature range (either through temperature fluctuation or a change in the alarm range by an operator), or
\item a failure is signalled in any of the input devices (temperature sensor and operator settings).
\end{mylist}~}
{The alarm temperature range is $\mv{al}\upto\mv{ah}$.

Monitored variable \mv{st} 
%in Table~\ref{fig:mv} 
shows ``invalid'' when any of the input signals fail.}
\label{R3}

\reqm{REQ}
{Once the alarm is activated, it becomes deactivated in one of two ways:
\begin{mylist}
\item The nurse turns off the Isolette;
\item The alarm has lasted for 10 seconds, and after 10 seconds or more the alarm conditions are removed.
\end{mylist}~\\}
{\hl{Refer to the relevant tables of monitored and/or controlled variables and function tables.}}
\label{R4}

\reqm{ENV}
{The current temperature received from the sensor is a a real number in the range $68.0$ to $105.0 \degree{F}$.\\}
{\hl{Refer to the relevant tables of monitored and/or controlled variables and function tables.}}
\label{E1}

\reqm{ENV}
{The desired and alarm temperatures received from the operator are all in increments of $1 \degree{F}$.\\}
{\hl{Refer to the relevant tables of monitored and/or controlled variables and function tables.}}
\label{E2}